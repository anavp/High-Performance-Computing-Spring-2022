\documentclass[12pt]{article}

%% FONTS
%% To get the default sans serif font in latex, uncomment following line:
 \renewcommand*\familydefault{\sfdefault}
%%
%% to get Arial font as the sans serif font, uncomment following line:
%% \renewcommand{\sfdefault}{phv} % phv is the Arial font
%%
%% to get Helvetica font as the sans serif font, uncomment following line:
% \usepackage{helvet}
\usepackage[small,bf,up]{caption}
\renewcommand{\captionfont}{\footnotesize}
\usepackage[left=1in,right=1in,top=1in,bottom=1in]{geometry}
\usepackage{graphics,epsfig,graphicx,float,subfigure,color}
\usepackage{amsmath,amssymb,amsbsy,amsfonts,amsthm}
\usepackage{url}
\usepackage{boxedminipage}
\usepackage[sf,bf,tiny]{titlesec}
 \usepackage[plainpages=false, colorlinks=true,
   citecolor=blue, filecolor=blue, linkcolor=blue,
   urlcolor=blue]{hyperref}
\usepackage{enumitem}
\usepackage{verbatim}
\usepackage{tikz,pgfplots}

\newcommand{\todo}[1]{\textcolor{red}{#1}}
% see documentation for titlesec package
% \titleformat{\section}{\large \sffamily \bfseries}
\titlelabel{\thetitle.\,\,\,}

\newcommand{\bs}{\boldsymbol}
\newcommand{\alert}[1]{\textcolor{red}{#1}}
\setlength{\emergencystretch}{20pt}

\begin{document}

\begin{center}
  \vspace*{-2cm}
{\small MATH-GA 2012.001 and CSCI-GA 2945.001, B.~Peherstorfer (Courant NYU); adapted from G.~Stadler}\end{center}
\vspace*{.5cm}
\begin{center}
\large \textbf{%%
Advanced Topics in Numerical Analysis: \\
High Performance Computing \\
Assignment 4 (due Apr.\ 18, 2022) }
\end{center}



\noindent {\bf Handing in your homework:} Hand in your homework as for
the previous homework assignments (git repo with Makefile), answering
the questions by adding a text or a \LaTeX\ file to your repo.
\\[.2ex]

% ****************************
\begin{enumerate}
% --------------------------

\item {\bf Matrix-vector operations on a GPU.} Write CUDA code for an inner product between two given
  (long) vectors on a GPU. Then, generalize this code to implement a
  matrix-vector multiplication (no blocking needed here) on the
  GPU. Check the correctness of your implementation by performing the same
  computation on the CPU and compare them. Report the memory band your
  code obtains
  on different GPUs.\footnote{The cuda\{1--5\}.cims.nyu.edu compute servers at
  the Institute have different Nvidia GPUs, for an overview see the list of
  compute servers available at the Institute:
  \url{https://cims.nyu.edu/webapps/content/systems/resources/computeservers}.}

\item {\bf 2D Jacobi method on a GPU.}
  Implement the 2D Jacobi method as discussed in the 2nd homework
  assignment using CUDA. Check the correctness of your implementation by performing the same
  computation on the CPU and compare them. 
  
  \item {\bf Update on final projection}
    Describe with a few sentences the status of your final project: What tasks that you formulated in the proposal have you worked on? Did you run into unforeseen issues? 

\end{enumerate}

\end{document}
